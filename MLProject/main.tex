\documentclass[twoside,a4paper,12pt]{report}
\usepackage{graphicx}
\usepackage{amsmath}
\usepackage{appendix}
\usepackage{nameref}
\usepackage{url}
\def\UrlBreaks{\do\/\do-}
\Urlmuskip=0mu plus 1mu
\usepackage[
backend=biber,
style=alphabetic
]{biblatex}
\setcounter{biburllcpenalty}{7000}
\setcounter{biburlucpenalty}{8000}
 \usepackage[T1]{fontenc} 
 \usepackage[final]{microtype}
\emergencystretch=1em
\graphicspath{ {images/} }
\usepackage[utf8]{inputenc}
\usepackage[spanish,english]{babel}



\usepackage{titlesec}

\setcounter{secnumdepth}{4}

\titleformat{\paragraph}
{\normalfont\normalsize\bfseries}{\theparagraph}{1em}{}
\titlespacing*{\paragraph}
{0pt}{3.25ex plus 1ex minus .2ex}{1.5ex plus .2ex}
 
\setlength{\parindent}{4em}
\setlength{\parskip}{1em}
\renewcommand{\baselinestretch}{1.3}

\usepackage{indentfirst}

\usepackage{mathptmx}


\usepackage{listings}
\usepackage{xcolor}
 
\definecolor{codegreen}{rgb}{0,0.6,0}
\definecolor{codegray}{rgb}{0.5,0.5,0.5}
\definecolor{codepurple}{rgb}{0.58,0,0.82}
\definecolor{backcolour}{rgb}{0.95,0.95,0.92}
 
\lstdefinestyle{mystyle}{
    backgroundcolor=\color{backcolour},   
    commentstyle=\color{codegreen},
    keywordstyle=\color{magenta},
    numberstyle=\tiny\color{codegray},
    stringstyle=\color{codepurple},
    basicstyle=\ttfamily\footnotesize,
    breakatwhitespace=false,         
    breaklines=true,                 
    captionpos=b,                    
    keepspaces=true,                 
    numbers=left,                    
    numbersep=5pt,                  
    showspaces=false,                
    showstringspaces=false,
    showtabs=false,                  
    tabsize=2
}
 
\lstset{style=mystyle}




\usepackage[]{geometry}
 \geometry{
 a4paper,
 total={170mm,257mm},
 inner=25mm,
 outer=25mm,
 top=25mm,
 bottom=25mm
 }
 
 



\usepackage{csquotes}

\addbibresource{references.bib}

\usepackage{titlesec}

\titleformat{\paragraph}
{\normalfont\normalsize\bfseries}{\theparagraph}{1em}{}
\titlespacing*{\paragraph}
{0pt}{3.25ex plus 1ex minus .2ex}{1.5ex plus .2ex}
 
\setlength{\parindent}{4em}
\setlength{\parskip}{1em}
\renewcommand{\baselinestretch}{1.3}

\usepackage{indentfirst}

\usepackage[]{geometry}
 \geometry{
 a4paper,
 total={170mm,257mm},
 inner=25mm,
 outer=25mm,
 top=25mm,
 bottom=25mm
 }
 
% \title{Machine learning and pattern recognition project}
\usepackage{csquotes}
\begin{document}

\tableofcontents

\newpage


\setcounter{chapter}{1}


\section{Dataset}

The dataset contains the information of red and white variants of the Portuguese "Vinho Verde" wine. It has been splitted into Test and Train sets containing 1822 and 1839 samples respectively. Moreover, there are two classes: good quality (value 1) and bad quality wine (value 0), where each has twelve attributes: 

DISTRIBUTION OF EACH ATTRIBUTES AND RANGE

\begin{itemize}
    \item fixed acidity
    \item volatile acidity
    \item citric acid
    \item residual sugar
    \item chlorides
    \item free sulfur dioxide
    \item total sulfur dioxide
    \item density
    \item pH
    \item sulphates
    \item alcohol
\end{itemize}

\section{Dimensionality Reduction}

\subsection{PCA - Principal Component Analysis}

Principal Component Analysis or PCA is a widely used technique for dimensionality reduction of the large data set. Reducing the number of components or features costs some accuracy and on the other hand, it makes the large data set simpler, easy to explore and visualize. Also, it reduces the computational complexity of the model which makes machine learning algorithms run faster.

\subsection{Use PCA on our dataset}
\textbf{Steps Involved in PCA}
\begin{enumerate}
    \item Standardize the data. (with mean =0 and variance = 1)
    \item Compute the Covariance matrix of dimensions.
    \item Obtain the Eigenvectors and Eigenvalues from the covariance matrix
    \item Sort eigenvalues in descending order and choose the top k Eigenvectors that correspond to the k largest eigenvalues (k will become the number of dimensions of the new feature subspace k≤d, d is the number of original dimensions).
    \item Construct the projection matrix W from the selected k Eigenvectors.
    \item Transform the original data set X via W to obtain the new k-dimensional feature subspace Y.
\end{enumerate}

Analyzing the eigenvalues obtained we can see how, with 4/5 dimensions we can obtain almost all of the information.
Even trying to draw the cumulative graph of the eigenvalues we can see the same thing.
-insert plot cumulative explained variance.png-
We can then try to apply PCA and see if the accuracy improves significantly.



\section{Classifiers}

\subsection{Without previous analysis}

In order to classify the data, different models have been taken into account : mutlivariate gaussian classifier, tied covariance, Naive Bayes and a k-fold approach. The results are shown in the table \ref{diffTypesclass}.
\begin{table}
\centering
 \begin{tabular}{||c c||} 
 \hline \hline
 \makecell{Classifier} & \makecell{Accuracy (\%)} \\
 \hline\hline
 Multivariate & 80.46  \\ 
 \hline
 Tied Covariance & 81.44\\
 \hline
 Naive Bayes & 78.98 \\
 \hline \hline
\end{tabular}
\label{diffTypesclass}
\caption{Different types of classifiers and their accuracy}
\end{table}

The precision of the models seems to be similar, however the tied covariance model has a greater precision so far with 81.44 \% and Naive Bayes performs the worst with 78.98 \%.The explanation of the poor Naive Bayes performance can be due to the fact that when we perform the diagonalization of the covariances, we do so by putting to zero the other elements of the matrix, hence inevitably losing information. This loss of information might be the explanation between of a worse performance than the other classifiers.  


\end{document}
